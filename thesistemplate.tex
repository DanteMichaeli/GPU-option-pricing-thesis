%%%%%%%%%%%%%%%%%%%%%%%%%%%%%%%%%%%%%%%%%%%%%%%%%%%%%%%%%%%%%%%%%%%%
%%%%%%%%%%%%%%%%%%%%%%%%%%%%%%%%%%%%%%%%%%%%%%%%%%%%%%%%%%%%%%%%%%%%
%%                                                                %%
%% An example for writting your thesis using LaTeX                %%
%% Original version by Luis Costa,  changes by Perttu Puska       %%
%% Support for Swedish added 15092014                             %%
%%                                                                %%
%% This example consists of the files                             %%
%%         thesistemplate.tex (versio 2.01)                       %%
%%         opinnaytepohja.tex (versio 2.01) (for text in Finnish) %%
%%         aaltothesis.cls (versio 2.01)                          %%
%%         kuva1.eps                                              %%
%%         kuva2.eps                                              %%
%%         kuva1.pdf                                              %%
%%         kuva2.pdf                                              %%
%%                                                                %%
%%                                                                %%
%% Typeset either with                                            %%
%% latex:                                                         %%
%%             $ latex opinnaytepohja                             %%
%%             $ latex opinnaytepohja                             %%
%%                                                                %%
%%   Result is the file opinnayte.dvi, which                      %%
%%   is converted to ps format as follows:                        %%
%%                                                                %%
%%             $ dvips opinnaytepohja -o                          %%
%%                                                                %%
%%   and then to pdf as follows:                                  %%
%%                                                                %%
%%             $ ps2pdf opinnaytepohja.ps                         %%
%%                                                                %%
%% Or                                                             %%
%% pdflatex:                                                      %%
%%             $ pdflatex opinnaytepohja                          %%
%%             $ pdflatex opinnaytepohja                          %%
%%                                                                %%
%%   Result is the file opinnaytepohja.pdf                        %%
%%                                                                %%
%% Explanatory comments in this example begin with                %%
%% the characters %%, and changes that the user can make          %%
%% with the character %                                           %%
%%                                                                %%
%%%%%%%%%%%%%%%%%%%%%%%%%%%%%%%%%%%%%%%%%%%%%%%%%%%%%%%%%%%%%%%%%%%%
%%%%%%%%%%%%%%%%%%%%%%%%%%%%%%%%%%%%%%%%%%%%%%%%%%%%%%%%%%%%%%%%%%%%

%% Uncomment one of these:
%% the 1st when using pdflatex, which directly typesets your document in
%% pdf (use jpg or pdf figures), or
%% the 2nd when producing a ps file (use eps figures, don't use ps figures!).
\documentclass[english,12pt,a4paper,pdftex,sci,utf8]{aaltothesis}
%\documentclass[english,12pt,a4paper,dvips]{aaltothesis}

%% To the \documentclass above
%% specify your school: arts, biz, chem, elec, eng, sci
%% specify the character encoding scheme used by your editor: utf8, latin1

%% Use one of these if you write in Finnish (see the Finnish template):
%%
%\documentclass[finnish,12pt,a4paper,pdftex,elec,utf8]{aaltothesis}
%\documentclass[finnish,12pt,a4paper,dvips]{aaltothesis}

\usepackage{graphicx}


%% Use this if you write hard core mathematics, these are usually needed
\usepackage{amsfonts,amssymb,amsbsy}

%% Use the macros in this package to change how the hyperref package below 
%% typesets its hypertext -- hyperlink colour, font, etc. See the package
%% documentation. It also defines the \url macro, so use the package when 
%% not using the hyperref package.
%%
%\usepackage{url}

%% Use this if you want to get links and nice output. Works well with pdflatex.
\usepackage[hidelinks]{hyperref}
\hypersetup{pdfpagemode=UseNone, pdfstartview=FitH,
  colorlinks=true,urlcolor=red,linkcolor=blue,citecolor=black,
  pdftitle={Default Title, Modify},pdfauthor={Your Name},
  pdfkeywords={Modify keywords}}


%% All that is printed on paper starts here
\begin{document}

%% Change the school field to specify your school if the automatically 
%% set name is wrong
 \university{aalto University}
 \school{School of Science}

%% Only for B.Sc. thesis: Choose your degree programme. 
\degreeprogram{Computer Science}
%%


%% Valitse yksi näistä kolmesta
%%
%% Choose one of these:
\univdegree{BSc}
%\univdegree{MSc}
%\univdegree{Lic}

%% Your own name (should be self explanatory...)
\author{Daniel Michaeli}

%% Your thesis title comes here and again before a possible abstract in
%% Finnish or Swedish . If the title is very long and latex does an
%% unsatisfactory job of breaking the lines, you will have to force a
%% linebreak with the \\ control character. 
%% Do not hyphenate titles.
%% 

\thesistitle{A Review of GPU Acceleration Techniques in Option Pricing Models}
\place{Espoo}

%% For B.Sc. thesis use the date when you present your thesis. 
%% 
%% Kandidaatintyön päivämäärä on sen esityspäivämäärä! 
\date{16.2.2025}

%% B.Sc. or M.Sc. thesis supervisor 
%% Note the "\" after the comma. This forces the following space to be 
%% a normal interword space, not the space that starts a new sentence. 
%% This is done because the fullstop isn't the end of the sentence that
%% should be followed by a slightly longer space but is to be followed
%% by a regular space.
%%
\supervisor{M.Sc.\ Henrik Lievonen} %{Prof.\ Pirjo Professori}

%% B.Sc. or M.Sc. thesis advisors(s). You can give upto two advisors in
%% this template. Check with your supervisor how many official advisors
%% you can have.
%%
%\advisor{Prof.\ Pirjo Professori}
%\advisor{D.Sc.\ (Tech.) Olli Ohjaaja}
\advisor{M.Sc.\ Henrik Lievonen}

%% Aalto logo: syntax:
%% \uselogo{aaltoRed|aaltoBlue|aaltoYellow|aaltoGray|aaltoGrayScale}{?|!|''}
%%
%% Logo language is set to be the same as the document language.
%% Logon kieli on sama kuin dokumentin kieli
%%
\uselogo{aaltoBlack}{''}

%% Create the coverpage
%%
\makecoverpage


%% Note that when writting your master's thesis in English, place
%% the English abstract first followed by the possible Finnish abstract

%% English abstract.
%% All the information required in the abstract (your name, thesis title, etc.)
%% is used as specified above.
%% Specify keywords
%%
%% Kaikki tiivistelmässä tarvittava tieto (nimesi, työnnimi, jne.) käytetään
%% niin kuin se on yllä määritelty.
%% Avainsanat
%%
\keywords{moi, mojn, moin, morjens, moro}
%% Abstract text
\begin{abstractpage}[english]

English bla bla bla
\end{abstractpage}

%% Force a new page so that the possible English abstract starts on a new page
%%
%% Pakotetaan uusi sivu varmuuden vuoksi, jotta 
%% mahdollinen suomenkielinen ja englanninkielinen tiivistelmä
%% eivät tule vahingossakaan samalle sivulle
\newpage
%

%% Force new page so that the Swedish abstract starts from a new page
\newpage
%
%% Swedish abstract. Delete if you don't need it. 
%% 
\thesistitle{Genomströmning och latens i datorsystem: en anlays av avvägningseffekter och optimeringsstrategier}
\advisor{M.Sc.\ Henrik Lievonen} %
\degreeprogram{Datateknik}
\department{Högskolan för teknikvetenskaper}%
\professorship{?}  %
%% Abstract keywords
\keywords{Nyckelord p\aa{} svenska,\\ Moi, moin, moidå, mojjdå}
%% Abstract text
\begin{abstractpage}[swedish]
 svenska bla bla bla
\end{abstractpage}

\newpage


%% Table of contents. 

\thesistableofcontents




%% Tweaks the page numbering to meet the requirement of the thesis format:
%% Begin the pagenumbering in Arabian numerals (and leave the first page
%% of the text body empty, see \thispagestyle{empty} below).
%% Additionally, force the actual text to begin on a new page with the 
%% \clearpage command.
%% \clearpage is similar to \newpage, but it also flushes the floats (figures
%% and tables).
%% There is no need to change these
%%
\cleardoublepage
\storeinipagenumber
\pagenumbering{arabic}
\setcounter{page}{1}


%% Text body begins. Note that since the text body
%% is mostly in Finnish the majority of comments are
%% also in Finnish after this point. There is no point in explaining
%% Finnish-language specific thesis conventions in English. Someday 
%% this text will possibly be translated to English.
%%
\section{Introduction}

%% Ensimm\"ainen sivu tyhj\"aksi
%% 
%% Leave first page empty
\thispagestyle{empty}
Options are a type of financial contract between two parties that grants the right - but not the obligation - to buy or sell a specific amount of an asset at a predetermined price by a specific future date \cite[p. 6]{hull2016options}. Options are a type of financial derivative, meaning that their value depends on the value of an underlying asset. Advances in both financial mathematics \cite{merton1994influence} and affordable computational power \cite{nordhaus2007two} have driven rapid growth in the derivatives market by improving pricing models and risk management tools. The use of derivatives has increased significantly in the last 50 years, not only among investors but also among so-called nonfinancial corporations \cite{bartram2009international}. The use of options and other derivatives allows for both levered speculation and sophisticated investment and risk management strategies by providing finer exposure control.

There are many different option pricing models, each with their pros and cons, and based on different underlying assumptions. Regardless of the choice of model, the importance of speed remains critical. All else being equal, faster computation leads to improved decision-making in investing and risk management because of either executing faster, having more data available, or both. This has led to the interest in re-implementing common option pricing models to run, either in part or fully, on the graphics processing unit (GPU) due to its powerful parallel processing capabilities compared to the central processing unit (CPU).

This thesis forms a literature review of common option pricing models' GPU acceleration potential and limitations. The aim is to assess how dependency structures in different models affect their GPU parallelization potential, what models see the greatest performance improvements of a GPU implementation, analyze the scalability of these solutions in terms of both time and space complexity, and identify related challenges and bottlenecks. For simplicity, the chosen models similar in nature have been grouped into three categories: lattice models, partial differential equation (PDE)-based models, and Monte Carlo (MC) methods. Section~\ref{sec:optionfundamentals} introduces formal definitions and the fundamental mathematics of option pricing. Section~\ref{sec:gpucomputing} presents a high-level overview of the GPU, how it differs architecturally from the CPU, and how it can be used for general-purpose computing. Sections ~\ref{sec:gpu-lattice}, ~\ref{sec:gpu-pde}, and ~\ref{sec:gpu-mc} review GPU implementation attempts of each respective model group, considering the aforementioned factors. The results are summarized in Section ~\ref{sec:summary} and conclusions are drawn in section ~\ref{sec:conclusions}.


For scope restriction purposes, the following choices have been made:
\begin{itemize}
    \item Only European and American style (vanilla) options are considered, with the exception of the Monte Carlo methods, which are particularly suitable for pricing exotic options with complex payoff structures. I will use the term "vanilla options" throughout the thesis to refer to both European and American style options (to be presented), whereas "exotic options" refer to more complex contracts. Unless explicitly stated otherwise, all discussions and conclusions apply to options in general. When specific terms are used, the statements pertain strictly to those contract styles.
    
    \item The purpose of this paper is to draw general conclusions about the GPU acceleration potential for different model groups. The referenced research uses varying microprocessor architectures and performance metrics. Other differences, like the number of active cores in the baseline CPU implementation, or the extent to which further optimization in e.g. memory access patterns have been implemented, also play a role. Thus, it must be acknowledged that direct quantitative comparisons are not always possible. Instead, I intend to incorporate the data into a more holistic, qualitative assessment that also considers the other factors of scalability and implementation difficulty.

    \item The models and implementations are studied purely from a computing performance perspective. That is, no interest is taken in the analysis of accuracy, conformity to assumptions, or any other unrelated aspect. The main comparisons of interest are the relative speedups of the GPU-accelerated implementations compared to a baseline solution, and to a smaller extent relative performance comparisons between different GPU-accelerated implementations.
\end{itemize}


%% Opinn\"aytteess\"a jokainen osa alkaa uudelta sivulta, joten \clearpage
%%
%% In a thesis, every section starts a new page, hence \clearpage
\clearpage

\section{Option Pricing Fundamentals} \label{sec:optionfundamentals}

The following section presents a sufficient theoretical introduction to options. Firstly, general definitions are established, after which 
vanilla option payoff functions are examined. Next, the risk-neutral valuation framework is presented, (non-rigorously) explaining the mathematical principles that underpin all major pricing models. This is followed by a subsection on the key determinants that influence option prices, and the sensitivity measures, known as "the Greeks", that quantify how prices change in response to these determinants. The section concludes with a discussion on the inherent computational challenges of option pricing, further motivating the exploration of GPU acceleration techniques.


\subsection{Vanilla Option Definitions and Payoff Structures}
Chapter 1.5 in \cite{hull2016options} presents the following useful definitions and related terminology:

\paragraph{Call Option:}Grants the holder the right (but not the obligation) to buy the underlying asset by a certain date for a certain price.

\paragraph{Put Option:}Grants the holder the right (but not the obligation) to sell the underlying asset by a certain date for a certain price.

\paragraph{Strike Price:}The predetermined price at which the underlying asset can be bought (for a call option) or sold (for a put option) upon exercise.

\paragraph{Maturity (date):}The predetermined date on which the option expires, determining the latest point at which it can be exercised.
\bigskip

Note the ambiguous use of "by", as the specific rules for when an option can be exercised depends on the style of option. For our vanilla options, a European-style option must be exercised on the maturity date, whereas an American-style option can be exercised at any time up to the maturity.

There are always two parties to every option contract: the investor (buyer), who takes the long position, pays an up front premium to the underwriter (seller) for the right to engage in a future trade. A single option contract is considered a zero-sum game between the investor and the underwriter. Thus, we have a pair of mirrored positions for both a call option and a put option. These can be visualized by graphing the profit of each position at the time of exercising as a function of the underlying asset price. Alternatively, one can consider the slightly modified \textbf{payoff} of the position, which ignores the premium and focuses on the fundamental mechanics of the option itself.

In similar fashion as in \cite[pp. 7-10]{hull2016options}, each position's profit diagram is presented below, along with a practical example for intuition. We consider European-style options, and transaction costs have been ignored.

\subsubsection{Call Option Positions}
Let $K$ be the strike price and $S_T$ the price of the underlying asset at maturity. The investor is willing to pay up front in order to fix a price for a later time. In other words, they expect the price of the underlying asset to rise enough to offset the premium. Figure 



As an example, consider a manufacturer that relies heavily on crude oil for their final product. If they expect a sharp increase in oil prices in a year from now they might choose to enter a long position in a call option with the oil distributor. The manufacturer knows it cannot afford the projected oil prices, and is willing to pay a premium in order to hedge against that risk.







\bigskip



- call vs put
- long vs short
    - always needs a matching pair?
- option = right / OPTION. differ vs other similar contracts with obligation like futures. Pay a premium for  Summary Diagram on long/short vs put/call, box chart of upsides and downsides for each position?
- Same for American, but they might be more worth due to option of exercising early!
- payoff versus
- Moneyness terms explanation with graphs, revers relationship between call and put, "intrinsic value" measure,

- Hull Chapters 1.5, 8, 9, 19 for exotics



\subsection{Risk-Neutral Valuation Framework}

\subsection{Option Price Determinants}

\subsection{The Greeks}

\subsection{Computational Challenges in Option Pricing}


\section{The GPU and Parallel Computing} \label{sec:gpu-computing}
\subsection{CPU vs. GPU Architecture}

Traditional CPU design philosophy
GPU architectural principles
SIMD vs. SIMT execution models
Memory hierarchy differences

\subsection{GPU Programming Model}

Thread hierarchy (threads, blocks, grids)
Memory spaces (global, shared, constant, texture)
Execution model and scheduling
CUDA/OpenCL programming paradigms

\subsection{Parallel Computing Fundamentals and GPU suitability}

Types of parallelism (data, task, instruction)
Amdahl's Law and theoretical speedup limits
Dependency chains and their impact
Synchronization requirements
Characteristics of "GPU-friendly" algorithms
Identifying parallelizable components
Common computation patterns in finance
Performance metrics and evaluation

\subsection{Performance Considerations} ???

Memory transfer bottlenecks
Thread divergence
Coalesced memory access
Occupancy optimization
Warp efficiency

\clearpage

\section{GPU Acceleration of Lattice Models} \label{sec:gpu-lattice}
\subsection{Binomial}
\subsection{Trinomial}
\subsection{???}

\section{GPU Acceleration of PDE Models} \label{sec:gpu-pde}
\subsection{Black-Scholes Model Framework}
\subsection{European option analytical solution GPU vs CPU}
\subsection{Finite Differences numerical method for??}
\subsection{???}

\section{GPU Acceleration of Monte Carlo Methods} \label{sec:gpu-mc}
\subsection{MC in general}
\subsection{MC in options' theory}
\subsection{??????}

T\"ass\"a osassa kuvataan k\"aytetty tutkimusaineisto ja
tutkimuksen metodologiset valinnat, sek\"a
kerrotaan tutkimuksen toteutustapa ja k\"aytetyt menetelm\"at. 

\clearpage


%% Huomaa seuraavassa kappaleessa lainausmerkkien ulkopuolella piste, 
%% koska piste ei lopeta lainattua tekstinp\"atk\"a\"a.
%% Jos lainattu tekstinp\"atk\"a loppuu v\"alimerkkiin, tulee v\"alimerkki
%% lainausmerkkien sis\"alle: 
%% "Et tu, Brute?" sanoi Caesar kuollessaan.
Tutkimustuloksien merkityst\"a on aina syyt\"a arvioida ja tarkastella
kriittisesti.  Joskus tarkastelu voi olla t\"ass\"a osassa, mutta se
voidaan my\"os j\"att\"a\"a viimeiseen osaan, jolloin viimeisen osan nimeksi
tulee >>Tarkastelu>>. Tutkimustulosten merkityst\"a voi arvioida my\"os
>>Johtop\"a\"at\"okset>>-otsikon alla viimeisess\"a osassa. 

T\"ass\"a osassa on syyt\"a my\"os arvioida tutkimustulosten luotettavuutta.
Jos tutkimustulosten merkityst\"a arvioidaan >>Tarkastelu>>-osassa,
voi luotettavuuden arviointi olla my\"os siell\"a. 

\clearpage

\section{Summary}  \label{sec:summary}

\section{Conclusions}  \label{sec:conclusions}


Opinn\"aytteen tekij\"a vastaa siit\"a, ett\"a opinn\"ayte on t\"ass\"a dokumentissa
ja opinn\"aytteen tekemist\"a k\"asittelevill\"a luennoilla sek\"a
harjoituksissa annettujen ohjeiden mukainen muotoseikoiltaan,
rakenteeltaan ja ulkoasultaan.\cite{grochowski2004best}


\cleardoublepage
\phantomsection

\addcontentsline{toc}{chapter}{\bibname}
\bibliographystyle{IEEEtran}
\bibliography{bibliography.bib}


\end{document}